%%%%
%BCBox用于定义,需要注明的结论,BrCBox用于应用计算,NorCBox默认格式用于普通的计算结果,GCBox用于列举,RCBox用于验证,PCBox用于证明
\documentclass{book}
%%%%页面调整
\usepackage{geometry}
\geometry{a4paper, scale=0.85}
%%%%
%\vspace{-2cm}用来调整标题、时间、作者的间距
\usepackage{color,amsmath,amssymb,bm,mathrsfs,gensymb}
%%%
\usepackage[dvipsnames,svgnames,x11names]{xcolor}
\usepackage{tcolorbox}%这两个包捆绑使用,可以用下面代码实现有背景色的box
%\tcblower把box分为上下两部分;\tcbline只是增加分割线
\tcbuselibrary{skins,breakable,theorems}%丰富tcolorbox包. skins丰富文本框;breakable实现自动分页的文本框;theorems生成定理类环境
\definecolor{HTColor-Shadow}{rgb}{0.98,0.94,0.88}%rgb最大值为255。
\definecolor{HTColor-Deep}{rgb}{0.78,0.62,0.52}
\definecolor{HTColor-DDeep}{rgb}{0.78,0.27,0.21}
\definecolor{NLColor-Shadow}{rgb}{0.75,0.85,0.90}
\definecolor{NLColor-Deep}{rgb}{0.46,0.63,0.72}
\definecolor{NLColor-DDeep}{rgb}{0.84,0.31,0.002}
\definecolor{XGColor-Shadow}{rgb}{0.95,0.85,0.75}
\definecolor{XGColor-Deep}{rgb}{0.94,0.57,0.39}
\definecolor{KLColor-Shadow}{rgb}{0.97,0.90,0.82}
\definecolor{KLColor-Deep}{rgb}{0.72,0.78,0.48}
\definecolor{XHColor-Shadow}{rgb}{0.97,0.87,0.86}
\definecolor{XHColor-Deep}{rgb}{0.56,0.56,0.78}
\newtcolorbox{PCBox}[2][]
{	colback=XHColor-Shadow,%背景颜色
	colframe=XHColor-Deep,%边框颜色
	%coltitle=black!90!white,%标题颜色
	colbacktitle = XHColor-Deep, %标题背景色
	%fonttitle = \bfseries,
	enhanced,
	attach boxed title to top left = {yshift = -2mm, xshift = 5mm},%标题左右移动
	boxed title style = {sharp corners},%标题左右移动
	breakable=true, %跨页
	title=#2,#1}
\newtcolorbox{GCBox}[2][]
{	colback=KLColor-Shadow,%背景颜色
	colframe=KLColor-Deep,%边框颜色
	%coltitle=black!90!white,%标题颜色
	colbacktitle = KLColor-Deep, %标题背景色
	%fonttitle = \bfseries,
	enhanced,
	attach boxed title to top left = {yshift = -2mm, xshift = 5mm},%标题左右移动
	boxed title style = {sharp corners},%标题左右移动
	breakable=true,
	title=#2,#1}
\newtcolorbox{RCBox}[2][]
{	colback=XGColor-Shadow,%背景颜色
	colframe=XGColor-Deep,%边框颜色
	%coltitle=black!90!white,%标题颜色
	colbacktitle = XGColor-Deep, %标题背景色
	%fonttitle = \bfseries,
	enhanced,
	attach boxed title to top left = {yshift = -2mm, xshift = 5mm},%标题左右移动
	boxed title style = {sharp corners},
	breakable=true,
	title=#2,#1}
\newtcolorbox{BCBox}[2][]
{	colback=NLColor-Shadow,%背景颜色
	colframe=NLColor-Deep,%边框颜色
	%coltitle=black!90!white,%标题颜色
	colbacktitle = NLColor-Deep, %标题背景色
	%fonttitle = \bfseries,
	%opacityframe=0,边框透明度
	%boxrule=0,%边框厚度
	enhanced,
	attach boxed title to top left = {yshift = -2mm, xshift = 5mm},%标题左右移动
	boxed title style = {sharp corners},
	breakable=true,
	title=#2,#1}
\newtcolorbox{BrCBox}[2][]
{	colback=HTColor-Shadow,%背景颜色
	colframe=HTColor-Deep,%边框颜色
	%coltitle=black!90!white,%标题颜色
	colbacktitle = HTColor-Deep, %标题背景色
	%fonttitle = \bfseries,
	enhanced,
	attach boxed title to top left = {yshift = -2mm, xshift = 5mm},%标题左右移动
	boxed title style = {sharp corners},%标题左右移动
	breakable=true,
	title=#2,#1}
\newtcolorbox{NorCBox}[1][]
{	notitle,
	top={#1},
		%colback=HTColor-Shadow,%背景颜色
		%colframe=HTColor-Deep,%边框颜色
		%coltitle=black!90!white,%标题颜色
		%colbacktitle = HTColor-Deep, %标题背景色
		%fonttitle = \bfseries,
		opacityframe=0.5,
		%enhanced,
		%attach boxed title to top left = {yshift = -2mm, xshift = 5mm},%标题左右移动
		%boxed title style = {sharp corners},%标题左右移动
		%title=#2,#1
		breakable=true,
	}
%%%
\usepackage{bookmark}%能用part分割
\usepackage{cancel}%大斜杠
\newcommand{\eref}[1]{Eq.~(\ref{#1})}
\newcommand{\efig}[1]{Fig.~\ref{#1}}
\newcommand{\etable}[1]{Table.~\ref{#1}}
%\newcommand{\ra}{\rangle}
%\newcommand{\la}{\langle}
\usepackage{diagbox}%\diagbox{1}{2}用来生成表格中的斜杠
\usepackage{caption}%minipage实现并列表格中的标题
\usepackage{cite}%引用文献,可以自动实现[3-50]这种引用
\renewcommand\bibname{Reference}%修改bib的名字;article类一般默认是reference,而book则不是;
\newcommand{\supcite}[1]{\textsuperscript{\cite{#1}}}%定义上指标的引用,但没办法读取bib文件
%%%%%
\usepackage{cases} %大括号
\usepackage{textcomp}%千分号\textperthousand, 摄氏度\textcelsius

\usepackage{graphicx,subfigure,float}	%图片与浮动
%\hspace{-2cm}放在\includefigure{}前面用来调整图片位置
\usepackage{geometry}	%设置页面
\geometry{a4paper}

\newtheorem{theorem}{Theorem}[section]	%定理	
\newtheorem{definition}{Definition}[section]	%定义
%\input{simplewick.sty}
\usepackage{hyperref}%实现跳转 \href{run:地址}{文字}可以实现链接本地文件
\usepackage{url}%PDF超链接
\hypersetup{
	colorlinks=true,%超链接是否带颜色
	linkcolor=black,%目录,图表等内部链接颜色
	citecolor=HTColor-DDeep,%修改cite的颜色
	runcolor=NLColor-DDeep,%运行本地文件超链接的颜色
	bookmarks=true,%生成书签
	bookmarksnumbered=true,%书签编号
	bookmarksopen=true%目录展开
	}
%%%%%缩减章节占用的空白
\usepackage{titlesec}
\titleformat{\chapter}[display]
{\bfseries\Huge}
{Chapter \, \thechapter}
{0pt}%Chapter和标题的距离
{}
\titlespacing{\chapter}
{0pt}%左右移动
{-50pt}%上下移动
{0pt}%离section的举例
%%%%%
\usepackage{listings}  %插入代码块,用lstlisting命令
\lstset{
	language = c++,
	backgroundcolor = \color{yellow!10},    % 背景色:淡黄
	basicstyle = \small\ttfamily,           % 基本样式 + 小号字体
	rulesepcolor= \color{gray},             % 代码块边框颜色
	breaklines = true,                  % 代码过长则换行
	numbers = left,                     % 行号在左侧显示
	numberstyle = \small,               % 行号字体
	keywordstyle = \color{blue},            % 关键字颜色
	commentstyle =\color{green!100},        % 注释颜色
	stringstyle = \color{red!100},          % 字符串颜色
	frame = shadowbox,                  % 用(带影子效果)方框框住代码块
	framexleftmargin=-1.7cm,				%伸缩代码块左边框
	xleftmargin=-1.5cm,					%改变代码块左边框的位置
	numbersep=-1.5cm,						%改变行号数字距离代码的相对位置
	tabsize=4,
	captionpos=t,
	showspaces = false,                 % 不显示空格
	columns = fullflexible,             % 字间距固定
	%escapeinside={<@}{@>}              % 特殊自定分隔符:<@可以自己加颜色@>
	morekeywords = {as},                % 自加新的关键字(必须前后都是空格)
	deletendkeywords = {compile}        % 删除内定关键字;删除错误标记的关键字用deletekeywords删!
}
%%%%%%%%%创建目录命令,把这些命令放到document中去
%\pdfbookmark{Book Cover}{title} 	%创建封面pdf标签;第一个{}参数是pdf标签的名字,第二个{}参数用于hyperlink,
%\maketitle							%显示封面
%\thispagestyle{empty}				%封面取消页码
%\frontmatter						%页码使用罗马字母
%\chapter*{Preface}					%前言
%\addcontentsline{toc}{chapter}{Preface}  %在目录中增加前言的跳转,在PDF中实现跳转
%\cleardoublepage				 	%消除一张空白页,为了在pdf中实现目录的跳转
%\pdfbookmark{\contentsname}{toc} 	%创建目录pdf标签
%\tableofcontents				  	%生成目录
%\mainmatter						%页码使用拉丁数字
%\include{正文}
%\appendix							  %增加附录
%\include{Appendix/Appendix}

%\bibliographystyle{xxx}			
	%声明文献格式。可调用的参数为
	%plain,按字母的顺序排列,比较次序为作者、年度和标题;
	%unsrt,样式同plain,只是按照引用的先后排序;
	%alpha,用作者名首字母+年份后两位作标号,以字母顺序排序;
	%abbrv,类似plain,将月份全拼改为缩写,更显紧凑;
%\bibliography{ref}					%调用文献bib,注意bib文件名和路径名全部不能有空格;用\cite引用文献
%在写bib的时候用\href{xx}实现网页的跳转;在哪里用\href,哪里就会变红;标准是为{\href{xxx}{标红文字}}
%\addcontentsline{toc}{chapter}{Reference} %PDF中增加Reference的跳转
%%%%%%%%%

%%%%%%%
\usepackage{simpler-wick,tikz-feynman} %Wick收缩
%关于wick的使用方法:代码如下
	%	
	%	\wick{
	%		\c1 k, \c2 k'   %c1,c2是第1,2个收缩的标志,如果只有一个收缩的话,两个收缩要在不同行。
	%		\vert
	%		\overline{\c1 \psi}_x \psi_x \overline{\psi}_y \c2 \psi_y  %overline 是比较长的上滑线,c1,c2是和这个元素收缩
	%		\vert
	%		p, p'
	%	}
	%	
%可能会用到的一些命令
%\textsuperscript{xx}			%用于实现4^th这种效果;
